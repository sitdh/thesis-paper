\section{Related Works}

Unit test is a testing process to locate errors in SUT by focusing 
only on the interested parts and eliminate interaction occurred 
between software components by creating a drive or stub 
\cite{Jorgensen2013}. In contrast, integration testing is a process 
to uncover errors that may occur even though all the components 
have been working properly together \cite{Jalote:1997:IAS:549018}; however, the test case 
for this process that has to cover all of the class interfaces 
is expensive, as there are a large number of interfaces between 
classes that must be considered by the software tester. 
V. Panthini and D. Prasad \cite{4425952} has proposed a generated test case 
based on a sequence diagram in order to identify interactions 
between objects. However, the sequence diagram may not reflect 
the current state of source code, due to the reason 
that source code might be changed to another appropriate development 
methodology or techniques. S. Z. Waheed and U. Qamar \cite{7339088} 
has proposed that the test case generation for the integration testing 
is based on the flow diagram data and selected DU 
paths that are used to be the test path.  However, software 
is a result of class communication and the test path 
must be as long as possible to traverse each component 
that used to work together.

According to the references given above, we found that there 
is not any approaches that generate a test case that can be traversed 
through selected test paths between objects which have been working 
together based on object-oriented development in order to cover 
all branch interfaces found in source code. In addition, 
we are confident that our proposal will come up with an appropriate 
data set for the future test case generations.
